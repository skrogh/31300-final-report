\section{Opgave 7}
\paragraph{ PI-lead }
En regulator designes ud fra metoden angivet i \emph{lecture\_9\_PID\_2.pdf}. Da
størrelsen af styresignalet som nævt ønskes så lille som muligt benyttes en
\emph{PI-lead} controller med lead leddet i tilbagekoblingssløjfen.

Alle udregninger sker i et automatiseret \emph{Matlab} script, vedhæftet i appendix
C

\subparagraph{Valg af fasemarginer:}
	En passende stor fasemargin vælges, så systemets styresignal $u$ ikke bliver
	for stort. En margin på $\gamma_m = 50^o$ vælges.
	Der reserveres $11^o$ til \emph{I}-leddet.
	$\alpha$ sættes til $0.1$, hvilket giver $54.9^o$ ekstra.
	
	Dette giver at der søges efter en fase på $180^o - 50^o - 11^o + 54.9^o =
	-173.9^o$ vha. \emph{Matlab} script.
\subparagraph{Valg af $\omega_c$:}
	$\omega_c$ bestemmes så: $\phase{h1u(\omega_c)} = -173.9^o$. Dette giver
	$\omega_c =  0.2972~rad/sek$
\subparagraph{Valg af $\tau_i$ og $\tau_d$:}
	$\tau_i$ vælges ved at søge efter et $\tau_i$, der nettop giver et fasebidrag
	på $-11^o$ ved $\omega_c$. Dette gøres igen vha. \emph{Matlab}. $\tau_i =
	17.2982$.
	
	$\tau_d$ vælges så fasebidraget på $54.9^o$ placeres nettop ved $\omega_c$.
	$\tau_d = \frac{1}{\omega_c \sqrt{\alpha}} = 10.6414$.
\subparagraph{Valg af $K_p$:}
	$K_p$ vælges nu så forstærkningen ved $\omega_c$ bliver 0.
	Dette giver $K_p = 29.5747$
	
	For at opnå DC gain på $1$ ganges referencen med $k_m$
	
\paragraph{ Sliding mode }
	``Fordi vi kan'' blev en sliding mode controller også designet, denne
	behandles i appendix B
