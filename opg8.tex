\section{Opgave 8}
Både det lineære og ulineære system testes med \emph{PI-lead} controlleren.
Der udføres test både med \emph{lead}-leddet i frem- og tilbagekoblingssløjfen.

Som input i testen benyttes:
\begin{gather*}
d(t) = \left\{\begin{array}{r c l}
1 & \mbox{for} & 200<t<300\\
0 & \mbox{ellers} & 
\end{array}\right.\\
h_{1, set}(t) = \left\{\begin{array}{r c l}
 1 & \mbox{for} & 400<t<500\\
-1 & \mbox{for} & 600<t<700\\
 0 & \mbox{ellers} & 
\end{array}\right.\\
\end{gather*}
Hvor $h_{1, set}$ er referencen for $h1$

Det ses af \ref{fig:lin_sys_12_6} at settling tiden for systemet ikke afhænger
væsentligt af hvor \emph{lead} leddet er, men stigetiden gør.

For det ulineære system gælder imidlertid at de store udsving på $u$ med
\emph{lead}-leddet i fremkoblingsløkken giver stort oversving og lang
indsvingningstid.

Forskellen på det lineære og ulineære system omkring arbejdspunktet består
hovedsageligvis i inputbegrænsningen, dette ses af \ref{fig:lin_ulin_12_6}

\paragraph{Resultat}
	Målet var at systemet med en step påvirkning på $2~cm^3/sek$ ikke oversteg
	$h_{10}$ med mere end $2~mm$ i mere end $20~sek$. Vores regulator overstiger
	med nævnte påvirkning $h_{10}$ med mere end $2~mm$ i under $15~sek$ og opfylder derved
	kravet. Zoom af \ref{fig:ulin_sys_12_6} omkring påvirkningen findes i
	\ref{fig:dist_zoom}



\stdfig{1}{lin_sys_12_6}{Simulering af det linære system med en \emph{PI-lead}
controller. \(h_1\) med lead leddet i tilbagekoblingssløjfen ses at opføre sig pænt i blå, mens
den med lead leddet i fremkoblingssløjfen har et betydeligt oversving. }{fig:lin_sys_12_6}

\stdfig{1}{ulin_sys_12_6}{Simulering af det ulinære system med en \emph{PI-lead}
controller. Her er \(h_1\) med lead leddet i fremkøblingssløjfen igen i grøn, men oversvinget er meget større.}{fig:ulin_sys_12_6}
