\section{Opgave 9}
	Systemet simuleres med referencen sat til $h_{1,set} = 100~cm$, både med
	\emph{lead}-leddet i frem- og tilbagekoblingen. Inputtet er det same som i
	\emph{Opgave 8}. Begyndelsesbetingelserne for både $h_1$ og $h_2$ sættes til
	$h_1(0) = h_2(0) = 100~cm$.
	
	Simuleringen er vist i \ref{fig:ulin_sys_100}. I de første $\approx 100~sek$
	ser man \emph{I}-leddet rampe op, så det passer med det nye arbejdspunkt.
	
\paragraph{Resultat}
	I \ref{fig:dist_zoom_100} er der zoomet ind på området, hvor påvirknignen
	starter. Vandstanden afviger først med mere end $2~mm$ for meget i $\approx
	18~sek$ og kort tid efter med $2~mm$ for lidt i $\approx
	17~sek$. Derefter er vandstanden inden for grænsen på $\pm 2~mm$.
	
	\stdfig{1}{ulin_sys_100}{Simulering af det ulinære system med en \emph{PI-lead}
	controller, hvor arbejdspunktet er sat til $100~cm$. Igen ses det at responset med 
lead leddet i tilbagekoblingssløjfen (i blå) er betydeligt bedre end responset hvor 
lead leddet er i fremkoblingssløjfen.}{fig:ulin_sys_100}
