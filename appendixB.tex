\section*{Appendix B - Sliding mode controller}
\newcommand{\sign}{\mbox{sign}}
Som nævnt designes også en sliding mode controller.
Dette gøres da en af egenskaberne ved en sliding mode controller er god respons
på forstyrrelser - det bevirker ydermere at de performer godt for ulineære
systemer.

En output baseret sliding mode controller benyttes med
\begin{equation*}
\sigma = y - ref
\end{equation*}
hvor $ref$ er styresignalet og $y$ er outputtet fra systemet af tanke ($y = h_1
k_m$)

Et \emph{lead}-led benyttes som tilnærmelse til differentiering, således at:
\begin{equation*}
\diff\sigma = y \frac{s}{0.1s + 1}
\end{equation*}

Controlleren vælges så:
\begin{equation*}
u = K - K ~\sign( \diff\sigma ) + a ~ \sign(\sigma)\sqrt{\abs{\sigma}}
\end{equation*}

Hvor $K$ bestemmer amplituden for relæfunktionen, og dermed det mulige
arbejdsområde, samt mulighed for at modvirke påvirkninger.

$a$ er med til at bestemme området hvor sliding mode kan opnåes.

$K$ og $a$ vælges til $K = 4~u_0$ $a=15$

For at komme tættere på et fysisk system begrænses relæ-skiftefrekvensen til
$10Hz$, dette medfører naturligvis at systemet bliver marginalt stabit.
Amplituden af oscilleringen er dog langt under de $2~mm$.

\paragraph{ Resultater }
Sliding mode controlleren simuleres og sammenlignes med \emph{PI-lead}
controlleren, hvor \emph{lead}-leddet er placeret i tilbagekoblingen. Der
benyttes samme inputs som tidligere.

Controlleren simuleres både omkring arbejdspunktet $h_{10} = 12.6~cm$ i
\ref{fig:sliding_12_6} og $h_{10} = 100~cm$ i \ref{fig:sliding_100}.
Ved $h_{10} = 12.6~cm$ er der ikke den store forskel på sliding mode og den
lineære controller, men ved $h_{10} = 100~cm$ ser man tydeligt at systemes
ulinearitet ikke har lige så stor indflydelse.

\paragraph{ Ulemper }
\subparagraph{ Chattering }
	Skal systemet implementeres fysisk er det et problem at switching frekvensen
	skal holdes meget højere end systemets responstid, for at undgå stor amplitude på
	oscilleringen (chattering)
\subparagraph{ Begrænset arbejdsområde }
	Da der i relæfunktionen sættes en øvre grænse for $u$, sættes der også en
	grænse for den maksimale værdi $h_1$ kan antage.


\stdfig{1}{sliding_12_6}{Simulering af det ulineære system med en
sliding mode controller simuleret omkring arbejdspunktet
$h_{10} = 12.6~cm$}{fig:sliding_12_6}
\stdfig{1}{sliding_100}{Simulering af det ulineære system med en
sliding mode controller simuleret omkring arbejdspunktet
$h_{10} = 100~cm$}{fig:sliding_100}

\stdfig{1}{sliding_sur_12_6}{Sliding flade for sliding mode
controlleren simuleret omkring arbejdspunktet $h_{10} =
12.6~cm$}{fig:slidin_sur_12_6}

\stdfig{1}{sliding_sur_100}{Sliding flade for sliding mode
controlleren simuleret omkring arbejdspunktet $h_{10} =
100~cm$}{fig:slidin_sur_100}

\paragraph{ Litteratur }
\emph{
Arie Levant:
Universal Single-Input–Single-Output (SISO)
Sliding-Mode Controllers With Finite-Time
Convergence
(2000)}

\emph{
Yuri Shtessel,
Christopher Edwards,
Leonid Fridman,
Arie Levant:
Sliding Mode Control
and Observation
(2010)}
